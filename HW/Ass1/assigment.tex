%
% LATEXBONES
%
\documentclass[a4paper,11pt,twoside]{article}
\usepackage{graphicx}
\usepackage{amsmath}
\usepackage[english]{babel}
\usepackage[applemac]{inputenc}
\usepackage[colorlinks,bookmarks=false,linkcolor=blue,urlcolor=blue]{hyperref}
\usepackage{subfigure}
\usepackage{here}
\usepackage{wrapfig}
\usepackage{fancyhdr}
\usepackage{dirtytalk}

%drow graph
\usepackage{fancybox}
\usepackage{tikz}
\usepackage{capt-of}

% print code
\usepackage{listings}
\usepackage{algorithm2e}
\usepackage{verbatim}

% push at the bottom
\newenvironment{bottompar}{\par\vspace*{\fill}}{\clearpage}

% landscape
\usepackage{pdflscape}

\paperheight=297mm
\paperwidth=210mm

\setlength{\textheight}{235mm}
\setlength{\topmargin}{-1.2cm} 

\setlength{\parindent}{0pt}

\setlength{\textwidth}{15cm}
\setlength{\oddsidemargin}{0.56cm}
\setlength{\evensidemargin}{0.56cm}

% quotes
\usepackage{framed}
\newcommand*{\signed}[1]{%
  \unskip\hspace*{1em plus 1fill}%
  \nolinebreak[3]\hspace*{\fill}\mbox{#1}
}

\pagestyle{plain}

% multiline equation -- call it with \begin{multline}
\usepackage{amsmath}


% --- equations ---
\def \be {\begin{equation}}
\def \ee {\end{equation}}
%\def \dd  {{\rm d}}m

% --- links ---
\newcommand{\mail}[1]{{\href{mailto:#1}{#1}}}
\newcommand{\ftplink}[1]{{\href{ftp://#1}{#1}}}







% ======= Document ======

%----------------------------------------------------------------------------------------
% HEADING SECTIONS
%----------------------------------------------------------------------------------------

% --- header ---
\fancyhead[L]{Applied Data Analysis}
\fancyhead[R]{Summary}

% no newpage after title
\let\endtitlepage\relax

% no enumeration of sections
\setcounter{secnumdepth}{0}

\begin{document}
\begin{titlepage} %Titre
\begin{center}
\newcommand{\HRule}{\rule{\linewidth}{0.5mm}} % Defines a new command for the horizontal lines, change thickness here
\center % Center everything on the page
 
 
 %----------------------------------------------------------------------------------------
% TITLE SECTION
%----------------------------------------------------------------------------------------




\begin{figure} [h] %----------- SubGraph ---------------------
\centerline{
\subfigure{\includegraphics[height = 2 cm]{./pic/EPFL.png}  }
} 
\end{figure}

\HRule \\[0.4cm]
{ \huge \bfseries MGT-482 Principles of Finance \\Assignment 1}\\[0.4cm] % Title of your document

\begin{minipage}[t]{0.4\textwidth}
\flushleft
Prof. Erwan Morellec
\end{minipage}
~
\begin{minipage}[t]{0.55\textwidth}
\flushright
Team: \\
Joachim Muth - \mail{joachim.muth@epfl.ch}\\
Andreas Bill - \mail{andreas.bill@epfl.ch}\\
Nicolas Roth - \mail{nicolas.roth@epfl.ch}\\
\end{minipage}
\begin{center}
\today
\end{center}
\HRule \\
 %----------------------------------------------------------------------------------------

\end{center}
\end{titlepage}



\pagestyle{fancy}
% ================ Introduction ==============
% ================ Ex 1 ==============
\section{Exercice 1}

Following the rule of {\bf time value of money}, the present value is calculated with the following formula:
\be
PV(CF) = CF \cdot discount factor
\ee

Then the present of value is calculated with

\be
PV = 10'000_{1 year} + 10'000_{3 years} = \frac{10'000}{1.07} + \frac{10'000}{(1.07)^3} = 17'508.773
\ee

% ================ Ex 2 ==============
\section{Exercice 2}

The investment is

\be
PV(costs) = 1'000 + 5'000_{2 years} = 1'000 + \frac{5'000}{1.02^2} = 5'805.84
\ee

\begin{multline}
PV(benefits) = 4'000_{1 year} + 4'000_{2 years} + 4'000_{3 years} = \\ \frac{4'000}{1.02} + \frac{4'000}{1.02^2} + \frac{4'000}{1.02^3} = 11'535.53
\end{multline}

Then the \textbf{Net Present Value} of this project is 

\be
NPV = PV(benefits) - PV(costs) = 11'535.53 - 5'805.84 = 5'729.69
\ee

The NPV is positive so we would undertake the project.


% ================ Ex 3 ==============
\section{Exercice 3}

First we convert CHF into US Dollars: $250'000 CHF * 0.9 = \$ 225'000$

The investment is

\be
PV(costs) = \$ 225'000 
\ee

The {\bf Present Value} of the benefit is
\be
PV(benefits) = 310'000_{1 year}= \frac{310'000}{1.05}= 295'238.09
\ee

Then the \textbf{Net Present Value} of this project is 

\be
NPV = PV(benefits) - PV(costs) = 295'238.09 - 250'000 = 45238.09
\ee

The NPV is positive so we would undertake the project.

% ================ Ex 4 ==============
\section{Exercice 4}

Under the {\bf Law of One Price}, securities that produce the same cash flows must have the same price

\be
P(A) + P(B) = P(C)
\ee

As $P(A) = \$ 80$ and $P(C) = \$ 180$, $P(B)$ must be equal to 

\be
P(B) = P(C) - P(A) = 180 - 80 = \$100
\ee

Following this rule, there is no arbitrage opportunity.

% ================ Ex 5 ==============
\section{Exercice 5}

\begin{multline}
NPV(A) = -150 + 50_{1year} + 75_{2year} + 100_{3year} = \\ -150 + \frac{50}{1.1} + \frac{75}{1.1^2}+ \frac{100}{1.1^3} = 32.56
\end{multline}

\begin{multline}
NPV(B) = -250 + 150_{1year} + 150_{2year} - 50_{3year} = \\ -250 + \frac{150}{1.1} + \frac{150}{1.1^2}+ \frac{-50}{1.1^3} = -27.23
\end{multline}

You should by project A but not B.

% ================ Ex 6 ==============
\section{Exercice 6}

First we calcule the present value of the loan we have to pay to the bank

\be
PV = \frac{1000}{1.05} + \frac{1000}{1.05^2} + \frac{1000}{1.05^3} = 2723.24
\ee

Then we calculate what should the bank ask you to pay at the end of the third year to have exactly the same present value

\be
 PV_{offer 1} == PV_{offer 2} = \frac{x}{1.05^3} = 3152.49
 \ee
 
 \be
x = 3152.49
\ee

Then you should not accept a payment of more than $3152.49$.


% ================ Ex 7 ==============
\section{Exercice 7}

From your 18th birthday to your 25th, there is 7 years where you can let the money.

\be
V_{7 years} = V_{now} * (1.08)^7 = 39'960 * (1.08)^7 = 68'484.41
\ee

a) You will have \$ 68'484.41 at your 25th birthday

Originally, 18 years ago, the value of this money was
\be
V_{0 year} = \frac{V_{now}}{(1.08)^{18}} = \frac{39'960}{(1.08)^{18}} = 9'999.95
\ee

Your parents originally put \$ 9'999.95 in the account (if they are normal people, I guess they actually put \$10'000).

% ================ Ex 8 ==============
\section{Exercice 8}

The couple's spendings are represented as an outgoing cashflow. The present value of a cash flow is calculated as follows:

\be 
PV=CF \cdot \mathrm{Discout Rate}
\ee

One can determine the following table (the time scale starts at 0 due to payables at the beginning of the year):

\begin{center}
	\begin{tabular}{|l|c|c|c|c|c|}
		\hline
		Time & 0 & 1 & 2 & 3 & ... \\
		\hline
		Cash flow [\$] & 10000 & $10000 \cdot 1.05$ & $1000 \cdot 1.05^2$ & $1000 \cdot 1.05^3$ & ... \\
		\hline
		Present value added [\$] &10000 & $\frac{10000 \cdot 1.05}{1.05}$&$\frac{10000 \cdot 1.05^2}{1.05^2}$&$\frac{10000 \cdot 1.05^3}{1.05^3}$ &... \\
		\hline
	\end{tabular}
\end{center}

Summing up all the present values, one gets the value of the initial investment of \textbf{\$13'000}.


\section{Exercice 9}

The table below describes the savings made each year thanks to the new machine: 

\begin{center}
	\begin{tabular}{|l|c|c|c|c|c|c|}
		\hline
		Time & 1 & 2 & 3 & ... & i & ...\\
		\hline
		Savings [\$]& 1000 & $\frac{1000}{1.02}$ & $\frac{1000}{1.02^2}$ & ... & $\frac{1000}{1.02^i}$ & ...\\
		\hline
	\end{tabular}
\end{center}

In this problem, the constant payment is called the perpetuity C. Th present value of a growing perpetuity can easily be determined with the following formula:

\be
 PV(\mathrm{Savings}) = \frac{C}{r-g} 
\ee

where $r=0.05$ is the interest rate and $g=-0.02$ is the growth rate. The final present value of my saving therefore is \textbf{\$14'285.7}.

\section{Exercice 10}

In the first part of the problem, we will use the present value formula for an annuity $C$ for $N$ periods with an interest rate $r$:

\be
PV=C \cdot \frac{1}{r} \cdot \left(1-\frac{1}{(1+r)^N}\right)\\
\ee
\be
300'000=C \cdot \frac{1}{0.07} \cdot \left(1-\frac{1}{(1+0.07)^{30}}\right)
\ee
\be
C= \frac{300'000 \cdot 0.07}{1-\frac{1}{(1.07)^{30}}}= 24'175.92
\ee

The annual payment will be \textbf{\$24'175.92}.
\\ \newline
The second part of the problem consists in determine the present value for 30 years if only \$23'500 can be afforded per year. We have:

\be
PV = 23500 \cdot \frac{1}{0.07} \cdot \left( 1 - \frac{1}{1.07^{30}} \right) = 29'1612.5
\ee

The difference compared to the first situation is $300'000-291'612.5=8'387.5$ representing the present value of the balloon payment, adding the interests for 30 years, we have $8'387.5 \cdot 1.07^{30}=63'848$
\\ \newline
The balloon payment in 30 years will be \textbf{\$63'848}.

\section{Exercice 11}

In this case we can use two different types of present values $PV_1$ and $PV_2$. Let $PV_1$ be the annuity with payments that grow at the rate $g$ when the discount rate is $r$. The calculation is as follows:
\be
PV_1=C_1 \cdot \frac{1}{r-g} \cdot \left(1-\left(\frac{1+g}{(1+r)}\right)^N\right)+C_1
\ee

It is important not to forget the first payment done today (at t=0). Now, let $PV_2$ be the present value of \$2 million kept in an account during 35 years. We have:

\be
PV_2=\frac{C_2}{r^N}
\ee

The trick in this exercise is to equalize both $PV_1$ and $PV_2$ to determine the initial payment to make.

\be
PV_1=PV_2
\ee

\be
C_1 \cdot \frac{1}{r-g} \cdot \left(1-\left(\frac{1+g}{1+r}\right)^N\right)+C_1=\frac{C_2}{(1+r)^N}
\ee

\be
	C_1 = \frac{C_2 \cdot \frac{1}{(1+r)^{N}}}{1 + \frac{1}{r-g} \left( 1 - \left( \frac{1+g}{1+r} \right)^{N} \right)}= \frac{2000000 \cdot \frac{1}{1.05^{35}}}{1 + \frac{1}{0.02} \left( 1 - \left( \frac{1.03}{1.05} \right)^{35} \right)} = 14222.3
\ee

I will need to put \textbf{\$14'222.3} today into my account to reach the \$2 million in 35 years.

\section{Exercice 12}

In this exercise, at time $t=0$, my grandmother spends \$200'000 (negative cash flow). Then at time $t=1$, she receives \$25'000(positive cash flow), at time $t=2$ \$25'000, and so on until she dies. The present value of her annuity is:

\be
PV=C \cdot \frac{1}{r} \cdot \left(1-\frac{1}{(1+r)^N}\right)=25'000 \cdot \frac{1}{0.05} \cdot \left(1-\frac{1}{1.05^N}\right)\\
\ee

Isolating N in the above relation we get and setting the upper limit of $PV$ to \$200'000 we get

\be
N \geq \log \left( \frac{1}{1 - \frac{200'000 \cdot 0.05}{25'000}} \right) = 10.46
\ee

In order to reach the desired amount, she must live at least \textbf{11 years} after the day she retired. 



%======= TABLEAU ===========
%\begin{center} %---------------Tab--------------
%\begin{tabular} {| c | c | c | c | c | c |}
%\hline
 %& & & & & $\\ \hline
%\end{tabular}
%\end{center}


%===========GRAPH================
%\begin{figure} %---------------------Graph---------------------------
%\begin{center}
%\includegraphics[width=12cm]{graph/ampli2} 
%\end{center}
%\caption{\em  \label{label}
%L�gende
%}
%\end{figure}


%========SUBGRAPH=======
%\begin{figure} [h] %----------- SubGraph ---------------------
%\centerline{
%\subfigure[ sublegend ] {\label{sfig:thetat} \includegraphics[width=7cm]{ graph/graph_convdt3 } }
%\subfigure[ sublegend ] {\label{sfig:thetafin} \includegraphics[width=7cm]{ graph/graph_convtfin } } 
%}
%\caption{\label{ label } 
%L�gende
%} 
%\end{figure}








\end{document} %%%% THE END %%%%
