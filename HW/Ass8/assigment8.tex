%
% LATEXBONES
%
\documentclass[a4paper,11pt,twoside]{article}
\usepackage{graphicx}
\usepackage{amsmath}
\usepackage[english]{babel}
\usepackage[applemac]{inputenc}
\usepackage[colorlinks,bookmarks=false,linkcolor=blue,urlcolor=blue]{hyperref}
\usepackage{subfigure}
\usepackage{here}
\usepackage{wrapfig}
\usepackage{fancyhdr}
\usepackage{dirtytalk}

%drow graph
\usepackage{fancybox}
\usepackage{tikz}
\usepackage{capt-of}

% print code
\usepackage{listings}
\usepackage{algorithm2e}
\usepackage{verbatim}

% push at the bottom
\newenvironment{bottompar}{\par\vspace*{\fill}}{\clearpage}

% landscape
\usepackage{pdflscape}

\paperheight=297mm
\paperwidth=210mm

\setlength{\textheight}{235mm}
\setlength{\topmargin}{-1.2cm} 

\setlength{\parindent}{0pt}

\setlength{\textwidth}{15cm}
\setlength{\oddsidemargin}{0.56cm}
\setlength{\evensidemargin}{0.56cm}

% quotes
\usepackage{framed}
\newcommand*{\signed}[1]{%
  \unskip\hspace*{1em plus 1fill}%
  \nolinebreak[3]\hspace*{\fill}\mbox{#1}
}

\pagestyle{plain}

% --- equations ---
\def \be {\begin{equation}}
\def \ee {\end{equation}}
%\def \dd  {{\rm d}}m

% --- links ---
\newcommand{\mail}[1]{{\href{mailto:#1}{#1}}}
\newcommand{\ftplink}[1]{{\href{ftp://#1}{#1}}}






% ======= Document ======

%----------------------------------------------------------------------------------------
% HEADING SECTIONS
%----------------------------------------------------------------------------------------

% --- header ---
\fancyhead[L]{Finance}
\fancyhead[R]{Assignment 8}

\let\endtitlepage\relax

\begin{document}
\begin{titlepage} %Titre
\begin{center}
\newcommand{\HRule}{\rule{\linewidth}{0.5mm}} % Defines a new command for the horizontal lines, change thickness here
\center % Center everything on the page
 
 
 %----------------------------------------------------------------------------------------
% TITLE SECTION
%----------------------------------------------------------------------------------------




\begin{figure} [h] %----------- SubGraph ---------------------
\centerline{
\subfigure{\includegraphics[height = 2 cm]{./pic/EPFL.png}  }
} 
\end{figure}

\HRule \\[0.4cm]
{ \huge \bfseries MGT-482 Principles of Finance \\Assignment 8}\\[0.4cm] % Title of your document

\begin{minipage}[t]{0.4\textwidth}
\flushleft
Prof. Erwan Morellec
\end{minipage}
~
\begin{minipage}[t]{0.55\textwidth}
\flushright
Team: \\
Joachim Muth - \mail{joachim.muth@epfl.ch}\\
Andreas Bill - \mail{andreas.bill@epfl.ch}\\
Nicolas Roth - \mail{nicolas.roth@epfl.ch}\\
\end{minipage}
\begin{center}
\today
\end{center}
\HRule \\
 %----------------------------------------------------------------------------------------

\end{center}
\end{titlepage}



\pagestyle{fancy}

% ================ Ex 1 ==============
\section{Exercice 1}

Following answer document shows the formulas used for the computations. All numerical calculations have been carried out in the Excel sheet. \\

In order to estimate the value of the firm with the WACC using the DCF apporoach, we will go through three main steps. First, we will compute the $r_{WACC}$, then the free cash flow and finally the terminal value. Let's compute $r_{WACC}$ using the bottom-up approach as discussed in the lecture. The following formal determines $\beta_U$ of each firm $i$. 

\be
	{\beta_{U,i}} = \frac{{\beta_{E,i}} + (1-\tau) \cdot {\beta_{D,i}} \cdot {}^{D_i}/_{E_i}}{1 + (1-\tau) \cdot {}^{D_i}/_{E_i}}
\ee

The total $\beta_U$ is then obtained summing up $\beta_{U,i}$ of each firm taking the weight into account. We have

\be
\beta_U = \sum_{i} w_i \cdot {\beta_{U,i}} = \sum_{i} w_i \cdot \frac{{\beta_{E,i}} + (1-\tau) \cdot {\beta_{D,i}} \cdot {}^{D_i}/_{E_i}}{1 + (1-\tau) \cdot {}^{D_i}/_{E_i}}
\ee

with $\beta_{D,i}=0$ for all firms. Following relation converts $\beta_U$ to $\beta_L$:

\be
	\beta_L = \beta_U + (1-\tau) \cdot (\beta_U - \beta_D) \cdot {}^D/_E
\ee

It is now simple to determine the cost of equity, dept and $r_{WACC}$.

\be
\mathrm{Equity}=E(R_E) = R_F + \beta_L \cdot [E(R_M) - R_F)]$$	
$$\mathrm{Dept}=\mathrm{After\ tax\ cost} = (R_F + \mathrm{def.\ spread}) \cdot (1-\tau)$$
$$r_{WACC} = R_E \cdot \left(1 - \frac{\mathrm{Dept}}{\mathrm{Dept + Equity}}\right) + R_D \cdot \frac{\mathrm{Dept}}{\mathrm{Dept + Equity}}
\ee

Now we can determine the free cash flows. The $\mathrm{EBIT}_i$ for each year $i$ is calculated as follows

\be
	\mathrm{EBIT}_i= \mathrm{growth\ rate}_i \cdot \mathrm{EBIT}_{i-1}	
\ee

We took into account the tax of 35\% to determine the after tax $\mathrm{EBIT}_i$. The depreciation follows the same behavior than the $\mathrm{EBIT}_i$. Finally, we can compute the working capital as follows
\be
\mathrm{WC}_i = \mathrm{EBIT}_i \cdot 10\%
\ee
The free cash flow (FCF) is determined as follows

\be
\mathrm{FCF}_i = \mathrm{EBIT}(1-\tau)_i - \mathrm{CAPEX}_i + \mathrm{Depreciation}_i - \Delta \mathrm{WC}_i
\ee

with  

\be
\mathrm{CAPEX}_i = \mathrm{EBIT}(1-\tau)_i \cdot \mathrm{Reinvestment\ rate}_i + \mathrm{Depreciation}_i  - \Delta \mathrm{WC}_i
\ee

Finally, we can compute the present value of each year as follows, with $y$ the year number.

\be
\mathrm{PV}=\frac{\mathrm{FCF}}{(1+\mathrm{WACC})^{y}}
\ee

The terminal value at year 10 can be obtained with

\be
\mathrm{Terminal\ value} = \frac{\mathrm{FCF}_{11}}{{r_{WACC,10}} - g_{10}}
\ee

Once all data has been entered in the Excel file, we found a value of the firm of \textbf{\$50'174.8} with the fixed 10\% leverage ratio value. In order to find the value maximizing capital structure, we simply used the Solver add-in of Excel and found an optimal leverage ratio value of \textbf{24.99\%} leading to a firm value of \textbf{\$54'910.3}. 



%======= TABLEAU ===========
%\begin{center} %---------------Tab--------------
%\begin{tabular} {| c | c | c | c | c | c |}
%\hline
 %& & & & & $\\ \hline
%\end{tabular}
%\end{center}


%===========GRAPH================
%\begin{figure} %---------------------Graph---------------------------
%\begin{center}
%\includegraphics[width=12cm]{graph/ampli2} 
%\end{center}
%\caption{\em  \label{label}
%L�gende
%}
%\end{figure}


%========SUBGRAPH=======
%\begin{figure} [h] %----------- SubGraph ---------------------
%\centerline{
%\subfigure[ sublegend ] {\label{sfig:thetat} \includegraphics[width=7cm]{ graph/graph_convdt3 } }
%\subfigure[ sublegend ] {\label{sfig:thetafin} \includegraphics[width=7cm]{ graph/graph_convtfin } } 
%}
%\caption{\label{ label } 
%L�gende
%} 
%\end{figure}








\end{document} %%%% THE END %%%%
